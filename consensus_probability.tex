\section{Consensus Probability}

\paragraph{Epoch consensus\\}

In the following the probability for the Hashgraph-algorithm to reach an Epoch consensus within $H$ events consecutive is estimated in a simple model as follows.\\

The network participants are given by the flowing parameters.
\begin{itemize}
	\item[$H$] is the number of event for a node to reach Epoch-consensus.
	\item[$M$] is the total number of nodes which are available for the network (This includes active and passive nodes, not prospect nodes). 
	\item[$n$] is the number of nodes which has not yet been seen by the current node. 
\end{itemize}

The probability node n to gossip node m can be described as:
\begin{equation}
P_G = \left . \frac{1}{M-1} \right|_{n \neq m}
\end{equation}

The number $H$ is define as more than $2/3$ of the number $M$.

\begin{equation}
H = floor \left( \frac{2}{3} M \right)
\end{equation}

The probability for gossip node $n$ to node $m$ in $N$ consecutive events can described as:
\begin{equation}
P_{n,m,N} = \left. P_{n,m}^N \right|_{N > 0}
\end{equation}

The probability for $m$ nodes to gossip to a node in same event slot can be expressed as:
\begin{equation}
P_{m} = P_{G}^m
\end{equation}
\lstset{language=bash, numbers=left, numberstyle=\tiny, stepnumber=1, numbersep=5pt, tabsize=4}%, 
\begin{lstlisting}
gossip [N=    11 strong=  5 bw=        171]
gossip [N=    31 strong=  9 bw=       1396]
gossip [N=   101 strong= 10 bw=      14918]
gossip [N=  1001 strong= 14 bw=    1498538]
gossip [N= 10001 strong= 20 bw=  149868334]
gossip [N=100001 strong= 24 bw=14996214212]
\end{lstlisting}
The probability for gossip $h$ different nodes in $h+1$ consecutive events can calculated as:
\begin{equation}
P_{h,i} = \sum_{i=0}^{N-1} \prod _{n=1}^{h+1} P_n
\end{equation}

\begin{equation}
X_n = \sum_{i=0}^{infinity} {P_n}^n
\end{equation}

A binary vector $S_{h,H}$ is defined as follow:

\begin{align}
S_{h,H} = [b_{0}, b_{1} .. b_{h-1}] ~ b_{i} \in \mathbb{B}\\
\text{where}\\
\sum S_{h,H} = H
\end{align}




\begin{equation}
P(n,r) = \frac{n!}{(n-r)!}
\end{equation}
