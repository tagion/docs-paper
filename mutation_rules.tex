\section{Mutation rules}\label{sec:mutation}

In this sections the algorithm of bill gene mutations is described.

\subsection{Mutation base}
A mutation base vector $R$ is generated as an UDR bit-vector
\begin{equation}
 R = [\mu_{0}, \mu_{1} .. \mu_{N-1}] ~ \mu_{i} \in \mathbb{B}
\end{equation}

\subsection{Population gene mutation}
From a number $M$ of gene vectors $T_j$ a population mutation gene $B$ is defined.
\begin{equation}
 B = [\beta_{0}, \beta_{1} .. \beta_{N-1}] ~ \beta_{i} \in \mathbb{B}
\end{equation}

For all 1's for each vector is summed, as follows.
\begin{equation}
 s_{i} = \sum_{j=0}^{M-1}  t_{j,i}, ~ t_{j,i} \in \mathbb{B}
\end{equation}
Where $s_i$ is the sums of 1's for bit $i$ for all vectors $T_j$ and $t_{j,i}$ is the bits in the $T_j$ vectors.

The bits in the population gene is defined as follows.
\begin{equation}
 \beta_{i} = 
  \begin{cases}
  1 & \text{if} ~ ( 2 \cdot s_i > M) \\
  \mu_i & \text{if} ~ ( 2 \cdot s_i = M) \\
  0 & \text{otherwise}     
 \end{cases}
\end{equation}
Where $mu_i$ is the mutation base for the population $M$.

\subsection{Production gene mutation}
From a gene pair $a$ and $b$ the production gene is defined as:
\begin{equation}
 \gamma_{i} = 
  \begin{cases}
  a_i & \text{if} ~ ( \mu_i = 0) \\
  b_i & \text{otherwise}     
 \end{cases}
\end{equation}
And $\mu_i$ is the mutation base of the production mutation.

\subsection{Transaction mutation}
The bill mutation rules is as follows.

\begin{enumerate}[{B}.1]
 \item A population gene $B$ is calculated for all inputs
 \item The genes of the outputs is production mutated with the epoch gene
\end{enumerate}

The epoch gene is generated for all the outputs as follows:
\begin{enumerate}[{P}.1]
 \item A population gene $P$ is calculated for all the transaction output genes
 \item The previous epoch gene $E$ is produced with $P$ to generate a new $E$ gene
\end{enumerate}

The transaction rewards lottery is selected based on the gene distance between the output gene and the current epoch gene.



