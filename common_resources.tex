\subsection{Common Resources} \label{sec:common_resources}
The Tagion network is a common resource, meaning that no state or private entity should own it because it serves a common purpose for the whole community and the users of the system. 

The resources are the actual network, and the governance mechanism all the IP (intellectual property) related to Tagion, which is listed below.

\subsubsection{Open Source code}
The source code is released under a GNU GPL or similar license and owned by the Tagion Foundation. The licensing is in process. Closed or commercial projects do need permission to use the source code.
At the moment the git repository is not opened. Preparations to open the repository are being made.
Before the open-source license is defined, and the open patent license has been given to the Tagion Foundation from I25S ApS, the repository cannot be opened. It is planned to open the project as soon as possible. People with legitimate reasons to verify the code can write to info@tagion.org and a copy of the source code can be shared after a signed NDA is in place.

\subsubsection{Patents}
EU patents are filed in the company name I25S ApS. The patents concern 1) a database system and 2) a network gossip protocol.
The database system, which is implemented as the DART in Tagion, is a distributed database system that can efficiently search and store data based on a cryptographic hash enabling the network to execute transactions in parallel thus promoting performance and scalability.
The gossip protocol is an efficient way to share information among all members of the network.

\paragraph{Open Licenses}
An open license is defined to be a free patent license given to the open-source project that cannot be revoked.
All open-source projects having the same type of license as the Tagion Project are as a rule of thumb granted a free license that cannot be revoked automatically.
Other open-source projects need to apply for an open license. The licenses cannot be revoked when first given.
The Tagion Foundation and project is given an open license by I25S. Projects that fork the Tagion code need an open license by default.

\subsubsection{“Tagion” Trademark}
An EU Trademark is registered for “Tagion” and owned by the Tagion Foundation including related internet domains. 
The reason for the trademark is to make sure that projects, which are forking the source and governance model, cannot call themselves Tagion XYC. It confuses the end-users and community, thus multiple projects with similar names should be avoided.
Tagion can protect itself from other projects or entities that wish to associate themselves with Tagion for malicious purposes.
