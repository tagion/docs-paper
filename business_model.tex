\section{Business Model}
The business model consists of two parts, namely incentives and fee payments. The incentives are given to the nodes for their work and fees are paid by the users for using the system. 

\paragraph{Money printing - incentives }
New money is added to the system when an epoch has been completed, and the DART has reached a consensus. The newly printed money is rewarded to one of the active nodes if it has successfully executed the epoch. \\
The reward winning node is selected via a UDR Lottery, which is seeded from the bull’s eye hash of the DART where the epoch was generated. \\
The amount is calculated by an economic protocol controlled by the economic governance, see \cref{subsec:economic_governance}. \\

\paragraph{Money burning - payment}
When a transaction is performed in the network, more fees are paid by the user initiating the transaction. The fees depend on storage, the transaction amount and the script execution load. The fees paid to the network are burned; thus, the amount is taken out of the money supply.
A storage fee is paid per bytes of the total sum of bytes of all outputs stored in the DART. \\
A transaction fee is paid as a fraction of the total Tagion amount of the input of the transaction script. \\
The execution fee is calculated per script instructions where each instruction is priced. \\
If the total Tagion amount of the output transaction script is less than a specified limit, the whole amount is burned and the transaction is not valid.
Fees for decentralised exchanges are described in \cref{sec:dex}.
